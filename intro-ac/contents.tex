\mode*

% Since this a solution template for a generic talk, very little can
% be said about how it should be structured. However, the talk length
% of between 15min and 45min and the theme suggest that you stick to
% the following rules:  

% - Exactly two or three sections (other than the summary).
% - At *most* three subsections per section.
% - Talk about 30s to 2min per frame. So there should be between about
%   15 and 30 frames, all told.

% XXX this lecture needs revision, especially the later part of it.

\section{Introduction}

\subsection{Authentication, authorization and access control}

\begin{frame}
  \begin{itemize}
    \item A policy specifies who is allowed to do what.

    \item Access control enforces operational security policies.
  \end{itemize}
\end{frame}

\begin{frame}
  \begin{definition}
    \begin{itemize}
      \item We have an active entity: a \emph{subject} (representing 
        a \emph{principal}).

      \item The subject tries to access an \emph{object} with some \emph{access 
          operation}.

      \item To protect this, there is a \emph{reference monitor} granting or 
        denying this access.
    \end{itemize}
  \end{definition}
\end{frame}

\begin{frame}
  \begin{definition}[Authentication]
    \begin{itemize}
      \item Principals make statements.

      \item Let \(s\) be a statement.

      \item Authentication answers \enquote{Who said \(s\)?} by stating 
        a principal.
    \end{itemize}
  \end{definition}

  \pause{}

  \begin{definition}[Authorization]
    \begin{itemize}
      \item Let \(o\) be an object.

      \item Authorization answers \enquote{Who is trusted to access \(o\)?} by 
        stating a (list of) principal(s).
    \end{itemize}
  \end{definition}
\end{frame}

\begin{frame}
  \begin{idea}[Reference monitor]
    \begin{itemize}
      \item The reference monitor requires authentication of principals to be 
        able to authorize the subject it represents.

      \item By authorization the reference monitor decides whether to grant or 
        deny a subjects request for access to an object.

      \item For this decision the reference monitor must use the security policy.
    \end{itemize}
  \end{idea}
\end{frame}


\subsection{Access operations}

\begin{frame}
  \begin{definition}
    \begin{itemize}
      \item The elementary access modes for operations are to \emph{observe} or 
        to \emph{alter} a resource.

      \item Different \emph{access operations} requires combinations of access 
        modes.
    \end{itemize}
  \end{definition}

  \pause{}

  \begin{definition}
    \begin{itemize}
      \item An \emph{access right} is a right to perform an access operation.

      \item \emph{Privileges} are sets of access rights.
    \end{itemize}
  \end{definition}
\end{frame}

\begin{frame}
  \begin{example}[BLP]
    \begin{itemize}
      \item The Bell-LaPadula (BLP) model has four access rights:
        \begin{itemize}
          \item Execute
          \item Read
          \item Append (blind write)
          \item Write
        \end{itemize}

      \item These rights requires the two different modes:
        \begin{description}
          \item[Execute] requires none.
          \item[Append] requires alter.
          \item[Read] requires observe.
          \item[Write] requires observe and alter.
        \end{description}
    \end{itemize}
  \end{example}
\end{frame}

% XXX can the examples be adjusted towards Web?

\begin{frame}
  \begin{example}[Reference monitor]
    \begin{itemize}
      \item In a multi-user OS, processes uses the open(2) system call to request 
        access.

      \item The OS makes sure no conflicting accesses are granted.

      \item Note that some things can be used without a direct request.

      \item E.g.\ the user doesn't need read permission to execute a program.

    \end{itemize}
  \end{example}
\end{frame}

\begin{frame}
  \begin{example}[UNIX-like systems]
    \begin{itemize}
      \item In UNIX-like systems we have three access operations:
        \begin{itemize}
          \item Read
          \item Write
          \item Execute
        \end{itemize}

      \item These are applied to both files and directories, but differently for 
        each.

      \item You can read from a file, or list the content of a directory.

      \item You can write contents to a file, or create or rename files in 
        a directory.

      \item You can execute the file, or you can search the directory.

      \item Operations on subdirectories or files are thus handled by access 
        operations to the parent directory.

    \end{itemize}
  \end{example}
\end{frame}

% XXX add example on databases
\begin{frame}
  \begin{example}
    Policies for creating and deleting files are expressed by
    \begin{itemize}
      \item access control on the directory in UNIX-like systems, but
      \item specific create and delete right in Windows.
    \end{itemize}
  \end{example}

  \pause{}

  \begin{example}
    Policies for defining security settings such as access rights are handled 
    by
    \begin{itemize}
      \item access control on the directory in UNIX-like systems, but
      \item could be handled by right like grant and revoke.
    \end{itemize}
  \end{example}
\end{frame}


%%%%%%%%%%%%%%%%%%%%%%

\begin{frame}
  \small
  \printbibliography{}
\end{frame}

