\mode*

% Since this a solution template for a generic talk, very little can
% be said about how it should be structured. However, the talk length
% of between 15min and 45min and the theme suggest that you stick to
% the following rules:  

% - Exactly two or three sections (other than the summary).
% - At *most* three subsections per section.
% - Talk about 30s to 2min per frame. So there should be between about
%   15 and 30 frames, all told.

% XXX this lecture needs revision, especially the later part of it.

\section{Introduction}

\subsection{Authentication, authorization and access control}

\begin{frame}
  \begin{definition}[Authentication]
    \begin{itemize}
      \item Principals make statements.

      \item Let \(s\) be a statement.

      \item Authentication answers \enquote{Who said \(s\)?} by stating 
        a principal.
    \end{itemize}
  \end{definition}

  \pause

  \begin{definition}[Authorization]
    \begin{itemize}
      \item Let \(o\) be an object.

      \item Authorization answers \enquote{Who is trusted to access \(o\)?} by 
        stating a (list of) principal(s).
    \end{itemize}
  \end{definition}
\end{frame}

\begin{frame}
  \begin{itemize}
    \item A policy specifies who is allowed to do what.

    \item Access control enforces operational security policies.
  \end{itemize}
\end{frame}

\begin{frame}
  \begin{definition}
    \begin{itemize}
      \item We have an active entity: a \emph{subject} (representing 
        a \emph{principal}).

      \item The subject tries to access an \emph{object} with some \emph{access 
          operation}.

      \item To protect this, there is a \emph{reference monitor} granting or 
        denying this access.
    \end{itemize}
  \end{definition}
\end{frame}

\begin{frame}
  \begin{idea}[Reference monitor]
    \begin{itemize}
      \item The reference monitor requires authentication of principals.

      \item By authorization the reference monitor decides whether to grant or 
        deny a subject's request for access to an object.

      \item For this decision the reference monitor must use the security policy.
    \end{itemize}
  \end{idea}
\end{frame}


\subsection{Access operations}

\begin{frame}
  \begin{definition}
    \begin{itemize}
      \item An \emph{access operation} is an operation a subject can perform on 
        an object.

      \item An \emph{access right} is a right to perform an access operation.

      \item \emph{Privileges} are sets of access rights.
    \end{itemize}
  \end{definition}
\end{frame}

\begin{frame}
  \begin{example}[Simple operations]
    \begin{description}
      \item[read] The subject is allowed to read the object.
      \item[write] The subject is allowed to write the object.
    \end{description}
  \end{example}

  \pause

  \begin{example}[Social network]
    \begin{itemize}
      \item Objects: shared photos in a social network.
      \item Subjects: users of the social network.
      \item Subjects with read access: friends.
      \item Subjects with write access: me.
    \end{itemize}
  \end{example}
\end{frame}


%%%%%%%%%%%%%%%%%%%%%%

\begin{frame}
  \small
  \printbibliography
\end{frame}

