\title{%
  Reference Monitors
}
\author{Daniel Bosk}
\institute[MIUN IKS]{%
  Department of Information and Communication Systems,\\
  Mid Sweden University, SE-851\,70 Sundsvall.
}

\begin{frame}
  \maketitle
\end{frame}

\mode*

\begin{abstract}
  The reference monitor enforces access control policies.
It requires care to implement a reference monitor.
To do this we need to explore what is the trusted computing base and enforcing 
access control on the lower layers in the system architecture.

Gollmann treats this area in Chapter 6 of his book 
\citetitle{Gollmann2011cs}~\cite{Gollmann2011cs}.
Exercises 6.1, 6.3 and 6.5 in~\cite{Gollmann2011cs} are recommended for your 
learning.

\end{abstract}


% XXX refmon lecture needs review, especially the later part of it

% Since this a solution template for a generic talk, very little can
% be said about how it should be structured. However, the talk length
% of between 15min and 45min and the theme suggest that you stick to
% the following rules:  

% - Exactly two or three sections (other than the summary).
% - At *most* three subsections per section.
% - Talk about 30s to 2min per frame. So there should be between about
%   15 and 30 frames, all told.


\section{Introduction}

\subsection{Enforcing Policies}

\begin{frame}
  \begin{itemize}
    \item We now have authentication and authorization.
    \item But how do we enforce these access controls?
    \item This is where reference monitors come in.
  \end{itemize}
\end{frame}

\subsection{Definitions}

\begin{frame}
  \begin{definition}[Trusted Computing Base]
    \begin{itemize}
      \item The totality of protection mechanisms within a system which is 
        responsible for enforcing a security policy.
      \item Consists of one or more components which together enforces the 
        policy.
      \item Its ability to enforce a policy depends on proper configuration of 
        its security mechanisms and those mechanisms themselves.
    \end{itemize}
  \end{definition}
\end{frame}

\begin{frame}
  \begin{definition}[Reference Monitor]
    Is an abstract concept referring to an abstract machine which mediates all 
    subjects' accesses to objects.
  \end{definition}

  \pause{}

  \begin{definition}[Security Kernel]
    \begin{itemize}
      \item Constitutes hardware, firmware, software of a Trusted Computing 
        Base (TCB) which implement the reference monitor concept.
      \item It must mediate all accesses, be protected from modification and be 
        verifiable as correct.
    \end{itemize}
  \end{definition}
\end{frame}

% XXX add figure for schematic of reference monitor
%\begin{frame}{Schematic of Reference Monitor}
%\end{frame}

\subsection{Placing the Reference Monitor}

\begin{frame}
  \begin{itemize}
    \item The RM could be implemented in hardware using the microprocessor.

    \item It could be implemented in the OS kernel, e.g.\ access control in 
      UNIX-like systems or Windows.

    \item It could be implemented in the services layer, e.g.\ database systems 
      or Java Virtual Machine.

    \item Finally, it could be implemented in the application layer, i.e.\ 
      security checks in the application code.

  \end{itemize}
\end{frame}


% XXX add application layer examples for refmon
\section{Operating Systems}

\subsection{OS Integrity}

\begin{frame}
  \begin{itemize}
    \item One of the tasks of the OS is to prevent unauthorized access to 
      different resources.

    \item What if the attacker could modify the OS\@?

    \item Hence we need protection for the OS, we need to maintain its 
      integrity.

  \end{itemize}
\end{frame}

\begin{frame}
  \begin{itemize}
    \item Now we have the problem that a user must be able to use the OS\@.

    \item But the user shouldn't be able to misuse the OS\@.

    \item To help us achieve this we have
      \begin{itemize}
        \item Modes of Operation, and
        \item Controlled Invocation (Restricted Privilege).
      \end{itemize}

    \item These can be applied on any layer, be it OS or application.

  \end{itemize}
\end{frame}

\subsection{Modes of Operation}

\begin{frame}
  \begin{itemize}
    \item We must be able to distinguish between what the OS executes for 
      itself and what it executes on behalf of the user.

    \item A mode bit is used to indicate which mode a system is currently in.

    \item Usually we use only two modes, \emph{user mode} and \emph{kernel 
      mode}.

    \item This way we can limit the possibility of execution.

  \end{itemize}
\end{frame}

\begin{frame}
  \begin{itemize}
    \item One problem we have now is to allow a user to invoke the privileged 
      operations in the operating system.

    \item Clearly just flipping the mode bit wouldn't work, that way the user 
      can do anything.

    \item So, we want to be able to flip the mode bit under certain 
      circumstances only -- and also flip it back before returning to the user.

    \item This is called \emph{controlled invocation}.
  \end{itemize}
\end{frame}

\subsection{Mechanisms at the Core}

\begin{frame}
  \begin{itemize}
    \item Placing mechanisms at the core will allow us higher level of 
      assurance.

    \item Security mechanisms can be bypassed from the layer below.

    \item A more complex system gives less assurance.

    \item Mechanisms at the core can decrease overheads which decrease 
      performance.
  \end{itemize}
\end{frame}


\section{Computer Architechture}

%\subsection{Overview}
%
%\begin{frame}
%\end{frame}

\subsection{CPU}

\begin{frame}
  \begin{itemize}
    \item Registers, such as
      \begin{itemize}
        \item program counter,
        \item stack pointer,
        \item status register (state information).
      \end{itemize}

    \item ALU which executes instructions.
  \end{itemize}
\end{frame}

\subsection{Memory}

\begin{frame}
  \begin{itemize}
    \item RAM
    \item ROM
    \item EPROM (erasable, programmable)
    \item WROM (write once)
  \end{itemize}
\end{frame}

\begin{frame}
  \begin{itemize}
    \item Volatile memory -- fades, not vanishes.
    \item Non-volatile.
  \end{itemize}
\end{frame}

\begin{frame}
  \begin{itemize}
    \item Divide memory into logical units, good for security but more 
      difficult.
    \item Divide memory into pages of equal length, efficient but more 
      difficult for access control.
  \end{itemize}
\end{frame}

\subsection{Interrupts}

\begin{frame}
  \begin{itemize}
    \item Uses the interrupt vector to see at what address to start execution, 
      where the interrupt handler is located.

    \item Can be pointed to some other code?

  \end{itemize}
\end{frame}


\section{Security Mechanisms}

\subsection{Relative Addressing}

\begin{frame}
  \begin{itemize}
    \item Use a base and a limit register to limit the address space.
  \end{itemize}
\end{frame}

\subsection{Function Codes}

\begin{frame}
  \begin{itemize}
    \item The Motorola 68000 supported function codes for all addresses.
    \item This system included separation of user data, user code, kernel data, 
      kernel code.
  \end{itemize}
\end{frame}


%%% REFERENCES %%%

\begin{frame}[allowframebreaks]
  \printbibliography{}
\end{frame}
