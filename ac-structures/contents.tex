\mode*

% Since this a solution template for a generic talk, very little can
% be said about how it should be structured. However, the talk length
% of between 15min and 45min and the theme suggest that you stick to
% the following rules:  

% - Exactly two or three sections (other than the summary).
% - At *most* three subsections per section.
% - Talk about 30s to 2min per frame. So there should be between about
%   15 and 30 frames, all told.

% XXX this lecture needs revision, especially the later part of it.

\section{Access control structures}

\subsection{Access control matrix}

\begin{frame}
  \begin{itemize}
    \item We can adapt two different focuses on the policy.

    \item The first being, ``What is a principal allowed to do?''

    \item The second, ``What may be done with an object?''

      \pause{}

    \item Which one is suitable depends on the application.

    \item E.g.\ an OS usually takes the second approach as its purpose is to 
      manage objects.

    \item E.g.\ applications like databases might focus on what different users 
      are allowed to do.
  \end{itemize}
\end{frame}

\begin{frame}
  \begin{itemize}
    \item The access control structure is used to store an implemented policy.

    \item This structure should help to express the policy.

    \item Access rights for each combination of subject and object should be 
      possible to define.

    \item The importance of the choice of structure is shown when the system 
      scales up.

  \end{itemize}
\end{frame}

\begin{frame}
  \begin{definition}[Access control matrix]
    \begin{itemize}
      \item \(S\) be the set of subjects,
      \item \(O\) the set of objects, and
      \item \(A\) the set of access operations.

        \pause{}

      \item \emph{Access control matrix}: \( M = \left( M_{so} \right)\), where 
        \(s\in S\) and \(o\in O\).
      \item Each entry \(M_{so}\subseteq A\) specifies the operations subject 
        \(s\) may perform on the object \(o\).
    \end{itemize}
  \end{definition}
\end{frame}

\begin{frame}
  \begin{remark}
    \begin{itemize}
      \item The access control matrix is an abstract concept.
      \item It's not very suitable for implementation.
    \end{itemize}
  \end{remark}
\end{frame}

\subsection{Capabilities and ACLs}

\begin{frame}
  \begin{itemize}
    \item Capabilities focuses on the subject.
    \item Access rights are stored with the subject.
    \item Capabilities are essentially the rows of the access control matrix.
    \item Subjects may grant rights to other subjects.
    \item Maybe even grant right to grant rights.

    \item How do you know who may access what?
    \item How do you revoka a capability?
  \end{itemize}
\end{frame}

\begin{frame}
  \begin{itemize}
    \item Focuses on the objects.
    \item Access rights are stored with the object.
    \item ACLs are essentially the columns of the access control matrix.

    \item How do you check access right of a specified subject?
  \end{itemize}
\end{frame}

\subsection{Ownership}

\begin{frame}
  \begin{itemize}
    \item Who sets the policies?

    \item The policy is the governing rules of who may access what.

    \item Who sets or is allowed to change the policy could be assigned to
      \begin{itemize}
        \item the owner of a resource.
          This is called \emph{discretionary} access control.

        \item a system wide policy decreeing who is allowed access or not.
          This is called \emph{mandatory} access control.

      \end{itemize}
  \end{itemize}
\end{frame}


\section{Comparing Security Attributes}

\subsection{Partial Orderings}

\begin{frame}
  \begin{itemize}
    \item Some resources in e.g.\ the University's Computer Science Department 
      can be accessed by all students, other only by students in a particular 
      class etc.

    \item Department creates groups ``All'' and ``DT116G'', ``DT145G'' and 
      ``DV026G''.

    \item The groups DT116G and All are of course related, DT116G is a subgroup 
      of All and should access everything All can access too.

    \item However, there is no such relation between DT116G and DT145G.

  \end{itemize}
\end{frame}

\begin{frame}
  \begin{itemize}
    \item We can use these comparisons for security policy decisions.

    \item Is the group of the subject requesting access a subgroup of the group 
      allowed access?

    \item These relationships have a corresponding mathematical construction 
      called partial ordering.

  \end{itemize}
\end{frame}

\begin{frame}
  \begin{definition}
    A \emph{partial ordering} \(\leq\) on a set \(L\) is a relation on 
    \(L\times L\) that is
    \begin{itemize}
      \item reflexive, \(\forall a\in L, a\leq a\),
      \item transitive, \(\forall a,b,c\in L, a\leq b\land b\leq c\implies 
        a\leq c\),
      \item antisymmetric, \(\forall a,b\in L, a\leq b\land b\leq a\implies 
        a=b\).
    \end{itemize}
    If \(a\leq b\), we say that \(a\) dominates \(b\).
  \end{definition}
\end{frame}

% XXX add formal examples of partial orderings

\subsection{Lattices of Security Levels}

\begin{frame}
  \begin{definition}
    A \emph{lattice} \((L,\leq)\) is a set \(L\) with a partial ordering 
    \(\leq\) such that for every two elements \(a,b\in L\) there exists
    \begin{itemize}
      \item an least upper bound \(u\in L\colon a\leq u, b\leq u\) and for all 
        \(v\in L\colon (a\leq v\land b\leq v) \implies u\leq v\).

      \item a greatest lower bound \(l\in L\colon l\leq a, l\leq b\) and for 
        all \(k\in L\colon (k\leq a\land b\leq b) \implies k\leq l\).
    \end{itemize}
  \end{definition}
\end{frame}

% XXX add formal example of lattices

% XXX add security example of lattices
%\subsection{Multi-Level Security}
%
%\begin{frame}
%\end{frame}


%%%%%%%%%%%%%%%%%%%%%%

\begin{frame}
  \small
  \printbibliography{}
\end{frame}

