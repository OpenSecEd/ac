Once you have authenticated users you can support access control --- and this 
is also one of the main reasons to authenticate them in the first place.
Access control aims at controlling who may access what and how they may access 
it.
There are different models and ways to implement access control.
Here we will give an overview of the possibilities.
In particular, the \acp{ILO} are that you are able to:
\begin{itemize}
  \item \emph{understand} the fundamental access control models --- \ac{DAC}, 
    \ac{acMAC}, \ac{RBAC} and \ac{ABAC} --- and their relations.
  \item \emph{evaluate} advantages and disadvantages of different access 
    control solutions.
  \item \emph{analyse} a situation and \emph{design} a proper access control 
    solution.
\end{itemize}

The reading material is Chapter 5, followed by Chapters 11 and 12, in 
\citetitle{Gollmann2011cs}~\cite{Gollmann2011cs}.
You are also recommended to read Anderson's treatment of the subject, he treats 
this in Chapters 4, 8, and 9 in 
\citetitle{Anderson2008sea}~\cite{Anderson2008sea}.
Finally, you should be able to do exercises 5.1 5.2, 5.5, 5.6, 5.8 and 5.9 
in~\cite{Gollmann2011cs}.
