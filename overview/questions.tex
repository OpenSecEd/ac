\question[3]
% tags: ac
% tags: E:C:A
Describe the relation between authentication and authorization?

\begin{solution}
  Authentication is about verifying a subject.
  Authorization is about what this subject is allowed to do according to the 
  security policy.
\end{solution}


\question[3]
% tags: ac
% tags: E:C:A
What is a reference monitor?

\begin{solution}
  All access requests pass through the reference monitor.
  The reference monitor implements the security policy.
  It requires authenticated subjects to determine if they are authorized to 
  perform the requested operation on the requested object.
\end{solution}


\question[2]
% tags: ac
% tags: E:C
What is discretionary access control?

\begin{solution}
  The owner (creator) of the object may set the access policy for the data 
  object.
  This is what is common in normal file systems.
\end{solution}


\question[2]
% tags: ac
% tags: E:C
What is mandatory access control?

\begin{solution}
  Mandatory access control sets the access policy for created objects based on 
  fixed rules in the system.
\end{solution}


\question[3]
% tags: ac
% tags: E:C:A
What is attribute-based access control (ABAC)?

\begin{solution}
  It's an access control model.

  It uses attributes in the security policy: e.g.\ identities, age limits, 
  times.
  This requires authenticated attributes.

  This is the most general access control model.
\end{solution}


\question[2]
% tags: ac
% tags: E:C
What is identity-based access control?

\begin{solution}
  This is the most common access control model.

  It's based on the identities of the subjects only.
  The subject's identities must be authenticated.
\end{solution}
