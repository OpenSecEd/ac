\title{%
  Authorisation and Access Control
}
\author{Daniel Bosk}
\institute[MIUN IKS]{%
  Department of Information and Communication Systems,\\
  Mid Sweden University, SE-851\,70 Sundsvall.
}
\date{\today}

\mode*

\begin{frame}
  \titlepage{}
\end{frame}

\begin{frame}
  \tableofcontents
\end{frame}

% Since this a solution template for a generic talk, very little can
% be said about how it should be structured. However, the talk length
% of between 15min and 45min and the theme suggest that you stick to
% the following rules:  

% - Exactly two or three sections (other than the summary).
% - At *most* three subsections per section.
% - Talk about 30s to 2min per frame. So there should be between about
%   15 and 30 frames, all told.

% XXX this lecture needs revision, especially the later part of it.

\section{Introduction}

\subsection{Authentication, authorization and access control}

\begin{frame}
  \begin{itemize}
    \item A policy specifies who is allowed to do what.

    \item Access control enforces operational security policies.
  \end{itemize}
\end{frame}

\begin{frame}
  \begin{definition}
    \begin{itemize}
      \item We have an active entity: a \emph{subject} (representing 
        a \emph{principal}).

      \item The subject tries to access an \emph{object} with some \emph{access 
          operation}.

      \item To protect this, there is a \emph{reference monitor} granting or 
        denying this access.
    \end{itemize}
  \end{definition}
\end{frame}

\begin{frame}
  \begin{definition}[Authentication]
    \begin{itemize}
      \item Principals make statements.

      \item Let \(s\) be a statement.

      \item Authentication answers \enquote{Who said \(s\)?} by stating 
        a principal.
    \end{itemize}
  \end{definition}

  \pause{}

  \begin{definition}[Authorization]
    \begin{itemize}
      \item Let \(o\) be an object.

      \item Authorization answers \enquote{Who is trusted to access \(o\)?} by 
        stating a (list of) principal(s).
    \end{itemize}
  \end{definition}
\end{frame}

\begin{frame}
  \begin{idea}[Reference monitor]
    \begin{itemize}
      \item The reference monitor requires authentication of principals to be 
        able to authorize the subject it represents.

      \item By authorization the reference monitor decides whether to grant or 
        deny a subjects request for access to an object.

      \item For this decision the reference monitor must use the security policy.
    \end{itemize}
  \end{idea}
\end{frame}


\subsection{Access operations}

\begin{frame}
  \begin{definition}
    \begin{itemize}
      \item The elementary access modes for operations are to \emph{observe} or 
        to \emph{alter} a resource.

      \item Different \emph{access operations} requires combinations of access 
        modes.
    \end{itemize}
  \end{definition}

  \pause{}

  \begin{definition}
    \begin{itemize}
      \item An \emph{access right} is a right to perform an access operation.

      \item \emph{Privileges} are sets of access rights.
    \end{itemize}
  \end{definition}
\end{frame}

\begin{frame}
  \begin{example}[BLP]
    \begin{itemize}
      \item The Bell-LaPadula (BLP) model has four access rights:
        \begin{itemize}
          \item Execute
          \item Read
          \item Append (blind write)
          \item Write
        \end{itemize}

      \item These rights requires the two different modes:
        \begin{description}
          \item[Execute] requires none.
          \item[Append] requires alter.
          \item[Read] requires observe.
          \item[Write] requires observe and alter.
        \end{description}
    \end{itemize}
  \end{example}
\end{frame}

% XXX can the examples be adjusted towards Web?

\begin{frame}
  \begin{example}[Reference monitor]
    \begin{itemize}
      \item In a multi-user OS, processes uses the open(2) system call to request 
        access.

      \item The OS makes sure no conflicting accesses are granted.

      \item Note that some things can be used without a direct request.

      \item E.g.\ the user doesn't need read permission to execute a program.

    \end{itemize}
  \end{example}
\end{frame}

\begin{frame}
  \begin{example}[UNIX-like systems]
    \begin{itemize}
      \item In UNIX-like systems we have three access operations:
        \begin{itemize}
          \item Read
          \item Write
          \item Execute
        \end{itemize}

      \item These are applied to both files and directories, but differently for 
        each.

      \item You can read from a file, or list the content of a directory.

      \item You can write contents to a file, or create or rename files in 
        a directory.

      \item You can execute the file, or you can search the directory.

      \item Operations on subdirectories or files are thus handled by access 
        operations to the parent directory.

    \end{itemize}
  \end{example}
\end{frame}

% XXX add example on databases
\begin{frame}
  \begin{example}
    Policies for creating and deleting files are expressed by
    \begin{itemize}
      \item access control on the directory in UNIX-like systems, but
      \item specific create and delete right in Windows.
    \end{itemize}
  \end{example}

  \pause{}

  \begin{example}
    Policies for defining security settings such as access rights are handled 
    by
    \begin{itemize}
      \item access control on the directory in UNIX-like systems, but
      \item could be handled by right like grant and revoke.
    \end{itemize}
  \end{example}
\end{frame}


\section{Access control structures}

\subsection{Access control matrix}

\begin{frame}
  \begin{itemize}
    \item We can adapt two different focuses on the policy.

    \item The first being, ``What is a principal allowed to do?''

    \item The second, ``What may be done with an object?''

      \pause{}

    \item Which one is suitable depends on the application.

    \item E.g.\ an OS usually takes the second approach as its purpose is to 
      manage objects.

    \item E.g.\ applications like databases might focus on what different users 
      are allowed to do.
  \end{itemize}
\end{frame}

\begin{frame}
  \begin{itemize}
    \item The access control structure is used to store an implemented policy.

    \item This structure should help to express the policy.

    \item Access rights for each combination of subject and object should be 
      possible to define.

    \item The importance of the choice of structure is shown when the system 
      scales up.

  \end{itemize}
\end{frame}

\begin{frame}
  \begin{definition}[Access control matrix]
    \begin{itemize}
      \item \(S\) be the set of subjects,
      \item \(O\) the set of objects, and
      \item \(A\) the set of access operations.

        \pause{}

      \item \emph{Access control matrix}: \( M = \left( M_{so} \right)\), where 
        \(s\in S\) and \(o\in O\).
      \item Each entry \(M_{so}\subseteq A\) specifies the operations subject 
        \(s\) may perform on the object \(o\).
    \end{itemize}
  \end{definition}
\end{frame}

\begin{frame}
  \begin{remark}
    \begin{itemize}
      \item The access control matrix is an abstract concept.
      \item It's not very suitable for implementation.
    \end{itemize}
  \end{remark}
\end{frame}

\subsection{Capabilities and ACLs}

\begin{frame}
  \begin{itemize}
    \item Capabilities focuses on the subject.
    \item Access rights are stored with the subject.
    \item Capabilities are essentially the rows of the access control matrix.
    \item Subjects may grant rights to other subjects.
    \item Maybe even grant right to grant rights.

    \item How do you know who may access what?
    \item How do you revoka a capability?
  \end{itemize}
\end{frame}

\begin{frame}
  \begin{itemize}
    \item Focuses on the objects.
    \item Access rights are stored with the object.
    \item ACLs are essentially the columns of the access control matrix.

    \item How do you check access right of a specified subject?
  \end{itemize}
\end{frame}

\subsection{Ownership}

\begin{frame}
  \begin{itemize}
    \item Who sets the policies?

    \item The policy is the governing rules of who may access what.

    \item Who sets or is allowed to change the policy could be assigned to
      \begin{itemize}
        \item the owner of a resource.
          This is called \emph{discretionary} access control.

        \item a system wide policy decreeing who is allowed access or not.
          This is called \emph{mandatory} access control.

      \end{itemize}
  \end{itemize}
\end{frame}


\section{Access control models}

\subsection{Identity-based access control}

\begin{frame}
  \begin{itemize}
    \item To more easily manage access control for many subjects and objects we 
      need another approach than above.

    \item The solution is to introduce intermediate levels of complexity.

  \end{itemize}
\end{frame}

\begin{frame}
  \begin{itemize}
    \item We might be able to use identity based acces control (IBAC).
    \item IBAC doesn't scale well.

    \item Thus we add groups to handle multiple principals at the same time, 
      e.g.\ a computer security class.

    \item This makes things easier.

  \end{itemize}
\end{frame}

\subsection{Role-based access control}

\begin{frame}
  \begin{itemize}
    \item Another approach is to use roles.

    \item A role is a collection of procedures assigned to users.

    \item At a first look it reminds a lot about groups.

    \item However, this is a more high-level way of handling access control.

  \end{itemize}
\end{frame}

\begin{frame}
  \begin{itemize}
    \item The procedures have more complex semantics than just read or write.

    \item They can only be applied to objects of given data types.

    \item E.g.\ transfering funds in a bank.

    \item RBAC is typically found at the application level.
  \end{itemize}
\end{frame}

\begin{frame}
  \begin{itemize}
    \item We can further have role hierarchies, i.e.\ relationships between 
      roles.

    \item E.g.\ we can have a teacher and a teaching assistant role, where the 
      teacher has all rights of the TA\@.

    \item Separation of duties is an important principle in security, i.e.\ 
      when the same subject isn't allowed to do two related operations.

    \item There can be static and dynamic policies for separation of duties.

  \end{itemize}
\end{frame}

\begin{frame}
  \begin{description}
    \item[Flat RBAC] Users are assigned to roles, permissions are assigned to 
      roles.
      Hence users get permissions via roles.

    \item[Hierarchical RBAC]
      Adds support for role hierarchies.

    \item[Constrained RBAC]
      Adds separation of duties.
  \end{description}
\end{frame}

\subsection{Attribute-based access control}

\begin{frame}
  \begin{definition}
    \begin{description}
      \item[Policy enforcement point] Inspects request and generates 
        authorization request for PDP.
      \item[Policy decision point, PDP] Evaluates requests against policies.
        Returns permit or deny.
      \item[Policy information point] Can be used by PDP to access attribute 
        databases.
    \end{description}
  \end{definition}
\end{frame}

\begin{frame}
  \begin{example}[Attributes]
    \begin{itemize}
      \item Subject attributes, e.g.\ age, clearance, department, role, \dots
      \item Action attributes, e.g.\ read, delete, write, \dots
      \item Object attributes, e.g.\ type, owner, classification, location, 
        \dots
      \item Contextual attributes, e.g.\ time, location, \dots
    \end{itemize}
  \end{example}
\end{frame}

\subsection{Protection Rings}

\begin{frame}
  \begin{itemize}
    \item Multics introduced protection rings.

    \item Low-level version of the high-level BLP.

    \item These are mainly implemented in hardware and used to protect 
      integrity.

    \item Access control is based on which rings the subject and object are in.

    \item E.g.\ 0 contains kernel, 1 contains OS functionality, 2 contains 
      utilities, and 3 is for user processes.

  \end{itemize}
\end{frame}


\section{Comparing Security Attributes}

\subsection{Partial Orderings}

\begin{frame}
  \begin{itemize}
    \item Some resources in e.g.\ the University's Computer Science Department 
      can be accessed by all students, other only by students in a particular 
      class etc.

    \item Department creates groups ``All'' and ``DT116G'', ``DT145G'' and 
      ``DV026G''.

    \item The groups DT116G and All are of course related, DT116G is a subgroup 
      of All and should access everything All can access too.

    \item However, there is no such relation between DT116G and DT145G.

  \end{itemize}
\end{frame}

\begin{frame}
  \begin{itemize}
    \item We can use these comparisons for security policy decisions.

    \item Is the group of the subject requesting access a subgroup of the group 
      allowed access?

    \item These relationships have a corresponding mathematical construction 
      called partial ordering.

  \end{itemize}
\end{frame}

\begin{frame}
  \begin{definition}
    A \emph{partial ordering} \(\leq\) on a set \(L\) is a relation on 
    \(L\times L\) that is
    \begin{itemize}
      \item reflexive, \(\forall a\in L, a\leq a\),
      \item transitive, \(\forall a,b,c\in L, a\leq b\land b\leq c\implies 
        a\leq c\),
      \item antisymmetric, \(\forall a,b\in L, a\leq b\land b\leq a\implies 
        a=b\).
    \end{itemize}
    If \(a\leq b\), we say that \(a\) dominates \(b\).
  \end{definition}
\end{frame}

% XXX add formal examples of partial orderings

\subsection{Lattices of Security Levels}

\begin{frame}
  \begin{definition}
    A \emph{lattice} \((L,\leq)\) is a set \(L\) with a partial ordering 
    \(\leq\) such that for every two elements \(a,b\in L\) there exists
    \begin{itemize}
      \item an least upper bound \(u\in L\colon a\leq u, b\leq u\) and for all 
        \(v\in L\colon (a\leq v\land b\leq v) \implies u\leq v\).

      \item a greatest lower bound \(l\in L\colon l\leq a, l\leq b\) and for 
        all \(k\in L\colon (k\leq a\land b\leq b) \implies k\leq l\).
    \end{itemize}
  \end{definition}
\end{frame}

% XXX add formal example of lattices

% XXX add security example of lattices
%\subsection{Multi-Level Security}
%
%\begin{frame}
%\end{frame}


%\section{Authorisation}
%
%\subsection{Principle of least privilege}
%
%\begin{frame}
%  \begin{itemize}
%    \item The main point with the ``principle of least privilege'' is to deny 
%      access to things you normally don't need.
%
%    \item Why should your webserver have read-permission to the system password 
%      database?
%
%    \item It will never attempt to read it anyway, so why bother?
%
%    \item Because you never know what might happen to your webserver 
%      communicating with the outside world -- it can be forced to anything, 
%      then the operating system must prevent it.
%
%  \end{itemize}
%\end{frame}
%
%\begin{frame}
%  \begin{itemize}
%    \item Why should we have ``drop table''-permissions for reading the 
%      password database?
%
%    \item A slip-up in any function accessing the database can then allow the 
%      attacker to do anything.
%
%    \item Otherwise, the attacker must find a vulnerabiblity to exploit in the 
%      only functions with write permissions to that particular table.
%
%  \end{itemize}
%\end{frame}
%
%\begin{frame}{Keys to the kingdom again}
%  \begin{itemize}
%    \item But what happens when all parts run as the same no-privilege user?
%
%    \item Then they have rights to each other, so we're back at it.
%
%    \item Separate different applications with different system users.
%
%    \item This problem cannot arise inside the database.
%
%    \item But if the webserver and database server runs as the same user, the 
%      webserver can just kill the database server and read the files directly 
%      from disc instead of using the API.
%
%  \end{itemize}
%\end{frame}
%
%\subsection{Access Control Lists (ACL)}
%
%\begin{frame}
%  \begin{itemize}
%    \item An ACL specifies which users should have which permissions.
%
%    \item It usually has a list of permissions; e.g.\ read, write.
%      Sometimes also append, insert etc.
%
%    \item Each user or role is entered with a certain set of permissions.
%
%    \item There are ACLs in both operating system (OS), for running processes 
%      and file systems, but also inside databases.
%
%  \end{itemize}
%\end{frame}
%
%\begin{frame}
%  \begin{itemize}
%    \item Apply the principle of least privilege.
%
%    \item This is usually not the case, most files are readable by everyone 
%      when created.
%
%    \item The same applies to databases, unsecure by default.
%
%  \end{itemize}
%\end{frame}
%
%\subsection{Centralised Authorisation Routines}
%
%\begin{frame}
%  \begin{itemize}
%    \item Use centralised functions in your application to manage 
%      authorisation, i.e.\ to enforce ACLs.
%
%    \item Centralisation in the application means less code, less code means 
%      less potential for bugs.
%
%    \item Don't copy-paste authorisation code to all functions, use one 
%      centralised function which handles the ACL.
%
%  \end{itemize}
%\end{frame}
%
%\begin{frame}
%  \begin{itemize}
%    \item The ACLs can be represented as an authorisation matrix.
%
%    \item I.e.\ a huge table listing all users, actions and resources along 
%      with whether the user is allowed.
%
%    \item This can be implemented in a database.
%
%    \item All parts of the webapplication must conform to this, use the 
%      centralised functions to enforce it.
%
%  \end{itemize}
%\end{frame}
%
%\begin{frame}
%  \begin{itemize}
%    \item Make sure to check both actions and resources.
%
%    \item Is this user allowed to edit personal settings?
%      Yes, he is, he's logged in.
%
%    \item Whose personal settings did the logged-in user edit?
%
%  \end{itemize}
%\end{frame}
%
%\begin{frame}
%  \begin{itemize}
%    \item When implementing the authorisation checks, watch out for ``positive 
%      authorisation''!
%
%    \item I.e.\ don't assume things will succeed, always write for default 
%      deny.
%
%    \item If your code is aborted, in any line, it should fail the 
%      authorisation.
%
%  \end{itemize}
%\end{frame}
%
%\subsection{Static Content}
%
%\begin{frame}
%  \begin{itemize}
%    \item Remember to protect files in the webserver's file system, e.g.\ PDFs 
%      and images if they are not publicly available.
%
%    \item Either set some ACLs for them to disallow fetching them using direct 
%      URLs.
%
%    \item Another possibility is to generate the content on the fly.
%
%  \end{itemize}
%\end{frame}
%
%\subsection{Don't trust the user}
%
%\begin{frame}
%  \begin{itemize}
%    \item Don't trust anything from the client-side.
%
%    \item Store all critical data on server-side, all access control related 
%      things should be on the server -- and check every time!
%
%  \end{itemize}
%\end{frame}
%

%%%%%%%%%%%%%%%%%%%%%%

\begin{frame}
  \small
  \printbibliography{}
\end{frame}

