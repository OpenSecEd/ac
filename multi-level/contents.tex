\mode*

% Since this a solution template for a generic talk, very little can
% be said about how it should be structured. However, the talk length
% of between 15min and 45min and the theme suggest that you stick to
% the following rules:  

% - Exactly two or three sections (other than the summary).
% - At *most* three subsections per section.
% - Talk about 30s to 2min per frame. So there should be between about
%   15 and 30 frames, all told.


\section{Multi-Level Security}

\subsection{The Bell-LaPadula model}

\begin{frame}
  \begin{itemize}
    \item WWII and the Cold War led the NATO-countries to develop joint 
      classification model.

    \item The classifications are an ordered set of labels:

      \begin{itemize}
        \item Unclassified
        \item Confidential
        \item Secret
        \item Top Secret
      \end{itemize}

      % XXX add reference for labels of classification
    \item EU also has Restricted.
  \end{itemize}
\end{frame}

\begin{frame}
  \begin{definition}
    \begin{itemize}
      \item An individual has \emph{clearance} for a security level.
    \end{itemize}
  \end{definition}

  \pause

  \begin{block}{Summary}
    \begin{itemize}
      \item An individual can read information classified to lower levels.
      \item Requires special people who can declassify.
      \item Different protection for each level.
    \end{itemize}
  \end{block}
\end{frame}

\begin{frame}
  \begin{definition}[Bell--LaPadula, BLP]
    \begin{description}
      \item[NRU] (No Read Up, simple security property)
        No process may read data from a higher level.
      \item[NWD] (No Write Down, *-property)
        No process may write data to a lower level.
    \end{description}
  \end{definition}
\end{frame}

\begin{frame}
  \begin{remark}
    \begin{itemize}
      \item Bell and LaPadula's security policy model published in 1973.
      \item Systems implementing it are \emph{multilevel secure (MLS)}.
      \item This is \emph{mandatory access control}.
    \end{itemize}
  \end{remark}
\end{frame}

%\begin{frame}{\insertsubsectionhead}
%  \begin{block}{Utökning av BLP}
%    Inför \emph{tranquility property}, finns två varianter:
%    \begin{description}
%      \item[Stark] Säkerhetsklassificeringar förändras aldrig under systemets 
%        gång.
%      \item[Svag] Säkerhetsklassificeringar förändras aldrig så att 
%        säkerhetspolicyn bryts.
%    \end{description}
%  \end{block}
%  \begin{itemize}
%    \item Anledningen till den svaga är \emph{principle of least privilege}.
%    \item Börja på lägsta säkerhetsnivån och öka allteftersom data med högre 
%      nivå används -- \emph{high watermark principle}.
%    \item Vad händer då här en ny fil skapas senare?
%    \item Det kommer att behövas en \emph{trusted subject} för av 
%      avklassificera.
%  \end{itemize}
%\end{frame}

\subsection{Covert channels}

\begin{frame}
  \begin{question}
    \begin{itemize}
      \item What if we write to an existing file on a higher level?
      \item What should happen?
    \end{itemize}
  \end{question}
\end{frame}

\begin{frame}
  \begin{example}[Military logistics]
    \begin{itemize}
      \item A military storage containing Top Secret equipment.
      \item A logistician has only Secret clearance.
      \item What should this logistician see?
    \end{itemize}
  \end{example}

  \pause

  \begin{remark}[Alternatives]
    \begin{enumerate}
      \item \enquote{This storage is full of Top Secret stuff.}

        \pause

      \item \enquote{There is nothing here.}

        \pause

      \item \enquote{This storage is full of uninteresting rubber boots.}
    \end{enumerate}
  \end{remark}
\end{frame}

\begin{frame}
  \begin{remark}[The covert channel problem]
    \begin{itemize}
      \item Processes at different levels share resourses.
    \end{itemize}
  \end{remark}

  \pause

  \begin{solution}
    \begin{itemize}
      \item Try to reduce the bandwidth.
    \end{itemize}
  \end{solution}
\end{frame}

\begin{frame}
  \begin{example}[The NRL pump]
    \begin{itemize}
      \item Buffers, randomizes timing of return of acknowledgements.
      \item This limits the bandwidth.
    \end{itemize}
  \end{example}
\end{frame}

\subsection{The Biba model}

\begin{frame}
  \begin{idea}[The Biba model]
    \begin{itemize}
      \item Focus on integrity instead of confidentiality.
      \item Essentially BLP upside-down.
      \item Analogy: Using data with larger error, yields larger error in 
        result.
    \end{itemize}
  \end{idea}
\end{frame}

\begin{frame}
  \begin{block}{The Biba model}
    \begin{itemize}
      \item No Write Up: Can read from higher levels, but not write.
      \item No Read Down: Can write to lower levels, but not read.
    \end{itemize}
  \end{block}

  \pause

  \begin{example}[LOMAC, Linux]
    \begin{itemize}
      \item A process was downgraded from high to low when receiving data from 
        network.
    \end{itemize}
  \end{example}

  \pause

  \begin{example}[Windows Vista]
    \begin{itemize}
      \item Windows Vista ran Internet Explorer on low.
      \item Downloading malware cannot compromise the system.
    \end{itemize}
  \end{example}
\end{frame}


%%% REFERENCES %%%

\begin{frame}[allowframebreaks]
  \printbibliography{}
\end{frame}
