\mode*

% Since this a solution template for a generic talk, very little can
% be said about how it should be structured. However, the talk length
% of between 15min and 45min and the theme suggest that you stick to
% the following rules:  

% - Exactly two or three sections (other than the summary).
% - At *most* three subsections per section.
% - Talk about 30s to 2min per frame. So there should be between about
%   15 and 30 frames, all told.


\section{Flernivåsäkerhet}

\begin{frame}{\insertsubsectionhead}
  \begin{itemize}
    \item Andra världskriget och Kalla kriget ledde NATO-länderna att utveckla 
      en gemensam klassificeringsmodell för känsliga dokument.
    \item Klassificeringarna utgör etiketter som är ordnade i nivåer; från 
      \emph{ej sekretessbelagd (Unclassified)} till \emph{förtrolig 
      (Confidential)}, \emph{hemlig (Secret)} och \emph{kvalificerat hemlig 
      (Top Secret)}.
    % XXX add reference for labels of classification
    \item EU har även \emph{begränsad (Restricted)}.
    \item En individ har \emph{behörighet (clearance)} för en säkerhetsnivå.
  \end{itemize}
\end{frame}
\begin{frame}{\insertsubsectionhead}
  \begin{itemize}
    \item En individ kan läsa information som är klassificerat till alla nivåer 
      upp till den nivå denne är behörig.
    \item Det krävs speciella personer för att avklassificera dokument.
    \item Till de olika nivåerna hör olika skyddsnivåer.
  \end{itemize}
\end{frame}
\begin{frame}{\insertsubsectionhead}
  \begin{block}{Bell--LaPadulamodellen (BLP)}
    \begin{description}
      \item[NRU] (Simple security property, No read up) Ingen process får läsa 
        data från en högre nivå.
      \item[NWD] (*-property, No write down) Ingen process får skriva data till 
        en lägre nivå.
    \end{description}
  \end{block}
  \begin{itemize}
    \item Bell och LaPadulas säkerhetsmodell publicerades 1973.
    \item System som implementerar den kallas oftast \emph{flernivåsäkra system 
      (multilevel secure, MLS)}.
    \item Information kan inte flöda nedåt, bara uppåt.
    \item System som implementerar denna typ av policy oberoende av användaren 
      sägs ha \emph{obligatorisk åtkomstkontroll (mandatory access control)}.
  \end{itemize}
\end{frame}
\begin{frame}{\insertsubsectionhead}
  \begin{block}{Utökning av BLP}
    Inför \emph{tranquility property}, finns två varianter:
    \begin{description}
      \item[Stark] Säkerhetsklassificeringar förändras aldrig under systemets 
        gång.
      \item[Svag] Säkerhetsklassificeringar förändras aldrig så att 
        säkerhetspolicyn bryts.
    \end{description}
  \end{block}
  \begin{itemize}
    \item Anledningen till den svaga är \emph{principle of least privilege}.
    \item Börja på lägsta säkerhetsnivån och öka allteftersom data med högre 
      nivå används -- \emph{high watermark principle}.
    \item Vad händer då här en ny fil skapas senare?
    \item Det kommer att behövas en \emph{trusted subject} för av 
      avklassificera.
  \end{itemize}
\end{frame}

\subsection{Bibamodellen}
\begin{frame}{\insertsubsectionhead}
%  \begin{block}{Bibamodellen}
%  \end{block}
  \begin{itemize}
    \item Vanligen kallad ''Bell--LaPadula upp-och-ned''.
    \item Används för integritet medan BLP används för konfidentialitet.
    \item Kan läsa data från högre nivåer men inte skriva till dem.
    \item Följaktligen används \emph{low watermark principle}.
    \item Lägre nivåer innehåller osäkra data och data baserat på dessa kan 
      därför inte vara mindre osäkra.
    \item LOMAC i Linux där en process nedgraderades från hög till låg då den 
      mottog data från nätverket.
    \item Windows införde därefter liknande system för att säkra bland annat 
      Internet Explorer.
  \end{itemize}
\end{frame}

\subsection{Covert channels}

\begin{frame}
  Vad händer om jag skriver till en fil som redan finns på högre 
  konfidentialitetsnivå?
\end{frame}

\begin{frame}{\insertsubsectionhead}{NRL-pump}
  \begin{itemize}
    \item Används för att begränsa bandbredden för hemliga kanaler.
  \end{itemize}
\end{frame}
\begin{frame}{\insertsubsectionhead}{Logistiksystem}
  \begin{itemize}
    \item Ett militärlager med hemlig utrustning.
    \item En logistiker som inte har behörighet för hemlig, vad ska denne se?
  \end{itemize}
\end{frame}


%%% REFERENCES %%%

\begin{frame}[allowframebreaks]
  \printbibliography{}
\end{frame}
