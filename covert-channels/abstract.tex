When looking at secure systems it is easy to assume they are safe just because 
the secret keys are not directly reachable.
This is not always true.
Even if the key storage is unreachable, there is some information that can be 
extracted anyway.
For instance the fact \emph{that} two principals are communicating, \emph{when} 
they are communicating, the time each operation takes to perform, etc., is not 
provided any confidentiality.
The information possible to extract from this is what is called side-channel 
information.

An overview of this area is provided in Chapters 17 and 23 of 
\cite{Anderson2008sea}.
An interesting paper on this topic is 
\citetitle{genkin2013rsa}~\cite{genkin2013rsa} where the authors extract RSA 
keys using acoustic side-channels, i.e.~they analyse the sound emitted by the 
electrical circuitry to find the computations done and hence derive the RSA key 
used.

There is another aspect of this too, namely covert channels.
Covert channels are channels over which communication can take place, even with 
limited bandwidth, despite the prohibition of this due to the security policy.

