\mode*

% Since this a solution template for a generic talk, very little can
% be said about how it should be structured. However, the talk length
% of between 15min and 45min and the theme suggest that you stick to
% the following rules:  

% - Exactly two or three sections (other than the summary).
% - At *most* three subsections per section.
% - Talk about 30s to 2min per frame. So there should be between about
%   15 and 30 frames, all told.

% XXX this lecture needs revision, especially the later part of it.

\section{Access control models}

\subsection{Identity-based access control}

\begin{frame}
  \begin{itemize}
    \item To more easily manage access control for many subjects and objects we 
      need another approach than above.

    \item The solution is to introduce intermediate levels of complexity.

  \end{itemize}
\end{frame}

\begin{frame}
  \begin{itemize}
    \item We might be able to use identity based acces control (IBAC).
    \item IBAC doesn't scale well.

    \item Thus we add groups to handle multiple principals at the same time, 
      e.g.\ a computer security class.

    \item This makes things easier.

  \end{itemize}
\end{frame}

\subsection{Role-based access control}

\begin{frame}
  \begin{itemize}
    \item Another approach is to use roles.

    \item A role is a collection of procedures assigned to users.

    \item At a first look it reminds a lot about groups.

    \item However, this is a more high-level way of handling access control.

  \end{itemize}
\end{frame}

\begin{frame}
  \begin{itemize}
    \item The procedures have more complex semantics than just read or write.

    \item They can only be applied to objects of given data types.

    \item E.g.\ transfering funds in a bank.

    \item RBAC is typically found at the application level.
  \end{itemize}
\end{frame}

\begin{frame}
  \begin{itemize}
    \item We can further have role hierarchies, i.e.\ relationships between 
      roles.

    \item E.g.\ we can have a teacher and a teaching assistant role, where the 
      teacher has all rights of the TA\@.

    \item Separation of duties is an important principle in security, i.e.\ 
      when the same subject isn't allowed to do two related operations.

    \item There can be static and dynamic policies for separation of duties.

  \end{itemize}
\end{frame}

\begin{frame}
  \begin{description}
    \item[Flat RBAC] Users are assigned to roles, permissions are assigned to 
      roles.
      Hence users get permissions via roles.

    \item[Hierarchical RBAC]
      Adds support for role hierarchies.

    \item[Constrained RBAC]
      Adds separation of duties.
  \end{description}
\end{frame}

\subsection{Attribute-based access control}

\begin{frame}
  \begin{definition}
    \begin{description}
      \item[Policy enforcement point] Inspects request and generates 
        authorization request for PDP.
      \item[Policy decision point, PDP] Evaluates requests against policies.
        Returns permit or deny.
      \item[Policy information point] Can be used by PDP to access attribute 
        databases.
    \end{description}
  \end{definition}
\end{frame}

\begin{frame}
  \begin{example}[Attributes]
    \begin{itemize}
      \item Subject attributes, e.g.\ age, clearance, department, role, \dots
      \item Action attributes, e.g.\ read, delete, write, \dots
      \item Object attributes, e.g.\ type, owner, classification, location, 
        \dots
      \item Contextual attributes, e.g.\ time, location, \dots
    \end{itemize}
  \end{example}
\end{frame}

\subsection{Protection Rings}

\begin{frame}
  \begin{itemize}
    \item Multics introduced protection rings.

    \item Low-level version of the high-level BLP.

    \item These are mainly implemented in hardware and used to protect 
      integrity.

    \item Access control is based on which rings the subject and object are in.

    \item E.g.\ 0 contains kernel, 1 contains OS functionality, 2 contains 
      utilities, and 3 is for user processes.

  \end{itemize}
\end{frame}


%\section{Authorisation}
%
%\subsection{Principle of least privilege}
%
%\begin{frame}
%  \begin{itemize}
%    \item The main point with the ``principle of least privilege'' is to deny 
%      access to things you normally don't need.
%
%    \item Why should your webserver have read-permission to the system password 
%      database?
%
%    \item It will never attempt to read it anyway, so why bother?
%
%    \item Because you never know what might happen to your webserver 
%      communicating with the outside world -- it can be forced to anything, 
%      then the operating system must prevent it.
%
%  \end{itemize}
%\end{frame}
%
%\begin{frame}
%  \begin{itemize}
%    \item Why should we have ``drop table''-permissions for reading the 
%      password database?
%
%    \item A slip-up in any function accessing the database can then allow the 
%      attacker to do anything.
%
%    \item Otherwise, the attacker must find a vulnerabiblity to exploit in the 
%      only functions with write permissions to that particular table.
%
%  \end{itemize}
%\end{frame}
%
%\begin{frame}{Keys to the kingdom again}
%  \begin{itemize}
%    \item But what happens when all parts run as the same no-privilege user?
%
%    \item Then they have rights to each other, so we're back at it.
%
%    \item Separate different applications with different system users.
%
%    \item This problem cannot arise inside the database.
%
%    \item But if the webserver and database server runs as the same user, the 
%      webserver can just kill the database server and read the files directly 
%      from disc instead of using the API.
%
%  \end{itemize}
%\end{frame}
%
%\subsection{Access Control Lists (ACL)}
%
%\begin{frame}
%  \begin{itemize}
%    \item An ACL specifies which users should have which permissions.
%
%    \item It usually has a list of permissions; e.g.\ read, write.
%      Sometimes also append, insert etc.
%
%    \item Each user or role is entered with a certain set of permissions.
%
%    \item There are ACLs in both operating system (OS), for running processes 
%      and file systems, but also inside databases.
%
%  \end{itemize}
%\end{frame}
%
%\begin{frame}
%  \begin{itemize}
%    \item Apply the principle of least privilege.
%
%    \item This is usually not the case, most files are readable by everyone 
%      when created.
%
%    \item The same applies to databases, unsecure by default.
%
%  \end{itemize}
%\end{frame}
%
%\subsection{Centralised Authorisation Routines}
%
%\begin{frame}
%  \begin{itemize}
%    \item Use centralised functions in your application to manage 
%      authorisation, i.e.\ to enforce ACLs.
%
%    \item Centralisation in the application means less code, less code means 
%      less potential for bugs.
%
%    \item Don't copy-paste authorisation code to all functions, use one 
%      centralised function which handles the ACL.
%
%  \end{itemize}
%\end{frame}
%
%\begin{frame}
%  \begin{itemize}
%    \item The ACLs can be represented as an authorisation matrix.
%
%    \item I.e.\ a huge table listing all users, actions and resources along 
%      with whether the user is allowed.
%
%    \item This can be implemented in a database.
%
%    \item All parts of the webapplication must conform to this, use the 
%      centralised functions to enforce it.
%
%  \end{itemize}
%\end{frame}
%
%\begin{frame}
%  \begin{itemize}
%    \item Make sure to check both actions and resources.
%
%    \item Is this user allowed to edit personal settings?
%      Yes, he is, he's logged in.
%
%    \item Whose personal settings did the logged-in user edit?
%
%  \end{itemize}
%\end{frame}
%
%\begin{frame}
%  \begin{itemize}
%    \item When implementing the authorisation checks, watch out for ``positive 
%      authorisation''!
%
%    \item I.e.\ don't assume things will succeed, always write for default 
%      deny.
%
%    \item If your code is aborted, in any line, it should fail the 
%      authorisation.
%
%  \end{itemize}
%\end{frame}
%
%\subsection{Static Content}
%
%\begin{frame}
%  \begin{itemize}
%    \item Remember to protect files in the webserver's file system, e.g.\ PDFs 
%      and images if they are not publicly available.
%
%    \item Either set some ACLs for them to disallow fetching them using direct 
%      URLs.
%
%    \item Another possibility is to generate the content on the fly.
%
%  \end{itemize}
%\end{frame}
%
%\subsection{Don't trust the user}
%
%\begin{frame}
%  \begin{itemize}
%    \item Don't trust anything from the client-side.
%
%    \item Store all critical data on server-side, all access control related 
%      things should be on the server -- and check every time!
%
%  \end{itemize}
%\end{frame}
%

%%%%%%%%%%%%%%%%%%%%%%

\begin{frame}
  \small
  \printbibliography{}
\end{frame}

