\mode*

% Since this a solution template for a generic talk, very little can
% be said about how it should be structured. However, the talk length
% of between 15min and 45min and the theme suggest that you stick to
% the following rules:  

% - Exactly two or three sections (other than the summary).
% - At *most* three subsections per section.
% - Talk about 30s to 2min per frame. So there should be between about
%   15 and 30 frames, all told.

% XXX this lecture needs revision, especially the later part of it.

\section{Access control models}

\subsection{Identity-based and role-based}

\begin{frame}
  \begin{definition}[Identity-based access control, IBAC]
    \begin{itemize}
      \item Access rights are defined in terms of objects and subject's 
        identity.
    \end{itemize}
  \end{definition}

  \pause

  \begin{example}
    \begin{itemize}
      \item Alice may read document~\(D\).
      \item Bob may write document~\(D\).
    \end{itemize}
  \end{example}
\end{frame}

\begin{frame}
  \begin{remark}
    \begin{itemize}
      \item Obviously, this doesn't scale well with many users to handle.
    \end{itemize}
  \end{remark}
\end{frame}

\subsection{Role-based access control}

\begin{frame}
  \begin{idea}
    \begin{itemize}
      \item Need to manage access control more easily for many subjects and 
        objects.

      \item The solution is to introduce intermediate levels of complexity.

        \pause

      \item We can add groups and
        \begin{itemize}
          \item map users to groups,
          \item map groups to objects.
        \end{itemize}
    \end{itemize}
  \end{idea}
\end{frame}

\begin{frame}
  \begin{remark}
    \begin{itemize}
      \item Role-based access control (RBAC) generalizes this idea.
    \end{itemize}
  \end{remark}
  
  \begin{definition}[RBAC]
    \begin{description}
      \item[Role assignment]: A subject can perform an access operation only if 
        the subject has selected or been assigned a role.

      \item[Role authorization]: A subject's active role must be authorized for 
        the subject.

      \item[Right authorization]: A subject can perform an access operation 
        only if it's authorized for the user's active role.
    \end{description}
  \end{definition}
\end{frame}

\begin{frame}
  \begin{remark}
    \begin{itemize}
      \item RBAC allows for more complex semantics than read and write.
      \item RBAC allows for data types, not only files.
    \end{itemize}
  \end{remark}
\end{frame}

\begin{frame}
  \begin{block}{Extensions}
    \begin{description}
      \item[Flat RBAC] Users are assigned to roles, permissions are assigned to 
        roles.

      \item[Hierarchical RBAC]
        Adds support for role hierarchies: Teacher, Teacher's Assistant (TA).

      \item[Constrained RBAC]
        Adds separation of duties: someone may not switch from Student to TA in 
        the same course.
    \end{description}
  \end{block}
\end{frame}

\subsection{Attribute-based access control}

\begin{frame}
  \begin{remark}
    \begin{itemize}
      \item Attribute-based access control (ABAC) generalizes RBAC.
    \end{itemize}
  \end{remark}

  \begin{idea}
    \begin{itemize}
      \item Has a recommended architecture.
      \item Attributes can be about anything.
      \item Policies are boolean formulas of attributes.
    \end{itemize}
  \end{idea}
\end{frame}

\begin{frame}
  \begin{definition}[Architecture]
    \begin{description}
      \item[Policy enforcement point, PEP] Inspects request and generates 
        authorization request for PDP.

      \item[Policy decision point, PDP] Evaluates requests against policies.
        Returns permit or deny.

      \item[Policy information point, PIP] Can be used by PDP to access 
        attribute databases.
    \end{description}
  \end{definition}
\end{frame}

\begin{frame}
  \begin{example}[Attributes]
    \begin{itemize}
      \item Subject attributes: e.g.\ age, clearance, department, role, \dots
      \item Action attributes: e.g.\ read, delete, write, \dots
      \item Object attributes: e.g.\ type, owner, classification, location, 
        \dots
      \item Contextual attributes: e.g.\ time, location, \dots
    \end{itemize}
  \end{example}
\end{frame}

\begin{frame}
  \begin{definition}[ABAC policy]
    Set of rules of the form:
    \begin{itemize}
      \item IF \(b(\text{attributes})\) THEN allow/deny action
    \end{itemize}
  \end{definition}
\end{frame}

\begin{frame}
  \begin{example}[ABAC policy]
    \begin{itemize}
      \item User~\(U\) can view document~\(D\) if \(D\) is in the same 
        department as \(U\).
      \item \(U\) can edit \(D\) if \(D\) is in draft mode.
      \item Deny access to \(D\) from foreign connections.
    \end{itemize}
  \end{example}
\end{frame}

\subsection{Discretionary and mandatory models}

\begin{frame}
  \begin{question}
    \begin{itemize}
      \item Who should set the access policy?
    \end{itemize}
  \end{question}
\end{frame}

\begin{frame}
  \begin{definition}[Discretionary Access Control, DAC]
    \begin{itemize}
      \item The owner of an object sets privileges for other subjects.
    \end{itemize}
  \end{definition}

  \pause

  \begin{example}[File system]
    \begin{itemize}
      \item Alice creates a file~\(f\).
      \item She sets that Bob can read from, but not write to, \(f\).
    \end{itemize}
  \end{example}
\end{frame}

\begin{frame}
  \begin{definition}[Mandatory Access Control, MAC]
    \begin{itemize}
      \item The policy is enforced by the system.
    \end{itemize}
  \end{definition}

  \pause

  \begin{example}[Military system]
    \begin{itemize}
      \item General Alice has access to files~\(f_0, f_1, \dotsc, f_n\).
      \item She writes the file~\(f\).
      \item Only those who can access \(f_0, \dotsc, f_n\) can access \(f\).
      \item Bob who has access to \(f_1, \dotsc, f_{n-1}\) is denied.
    \end{itemize}
  \end{example}
\end{frame}

\begin{frame}
  \begin{remark}
    \begin{itemize}
      \item Since Alice has read \(f_0, \dotsc, f_n\), then \(f = F(f_0, 
          \dotsc, f_n)\).
      \item \Ie, there might be some trace of \(f_n\) in \(f\) --- which Bob is 
        not allowed to access.
    \end{itemize}
  \end{remark}
\end{frame}

\begin{frame}
  \begin{exercise}
    \begin{itemize}
      \item What kind of model do you prefer on a social network?
      \item Discretionary or mandatory?
      \item What would these two models look like in that setting?
    \end{itemize}
  \end{exercise}
\end{frame}


%\section{Authorisation}
%
%\subsection{Principle of least privilege}
%
%\begin{frame}
%  \begin{itemize}
%    \item The main point with the ``principle of least privilege'' is to deny 
%      access to things you normally don't need.
%
%    \item Why should your webserver have read-permission to the system password 
%      database?
%
%    \item It will never attempt to read it anyway, so why bother?
%
%    \item Because you never know what might happen to your webserver 
%      communicating with the outside world -- it can be forced to anything, 
%      then the operating system must prevent it.
%
%  \end{itemize}
%\end{frame}
%
%\begin{frame}
%  \begin{itemize}
%    \item Why should we have ``drop table''-permissions for reading the 
%      password database?
%
%    \item A slip-up in any function accessing the database can then allow the 
%      attacker to do anything.
%
%    \item Otherwise, the attacker must find a vulnerabiblity to exploit in the 
%      only functions with write permissions to that particular table.
%
%  \end{itemize}
%\end{frame}
%
%\begin{frame}{Keys to the kingdom again}
%  \begin{itemize}
%    \item But what happens when all parts run as the same no-privilege user?
%
%    \item Then they have rights to each other, so we're back at it.
%
%    \item Separate different applications with different system users.
%
%    \item This problem cannot arise inside the database.
%
%    \item But if the webserver and database server runs as the same user, the 
%      webserver can just kill the database server and read the files directly 
%      from disc instead of using the API.
%
%  \end{itemize}
%\end{frame}
%
%\subsection{Access Control Lists (ACL)}
%
%\begin{frame}
%  \begin{itemize}
%    \item An ACL specifies which users should have which permissions.
%
%    \item It usually has a list of permissions; e.g.\ read, write.
%      Sometimes also append, insert etc.
%
%    \item Each user or role is entered with a certain set of permissions.
%
%    \item There are ACLs in both operating system (OS), for running processes 
%      and file systems, but also inside databases.
%
%  \end{itemize}
%\end{frame}
%
%\begin{frame}
%  \begin{itemize}
%    \item Apply the principle of least privilege.
%
%    \item This is usually not the case, most files are readable by everyone 
%      when created.
%
%    \item The same applies to databases, unsecure by default.
%
%  \end{itemize}
%\end{frame}
%
%\subsection{Centralised Authorisation Routines}
%
%\begin{frame}
%  \begin{itemize}
%    \item Use centralised functions in your application to manage 
%      authorisation, i.e.\ to enforce ACLs.
%
%    \item Centralisation in the application means less code, less code means 
%      less potential for bugs.
%
%    \item Don't copy-paste authorisation code to all functions, use one 
%      centralised function which handles the ACL.
%
%  \end{itemize}
%\end{frame}
%
%\begin{frame}
%  \begin{itemize}
%    \item The ACLs can be represented as an authorisation matrix.
%
%    \item I.e.\ a huge table listing all users, actions and resources along 
%      with whether the user is allowed.
%
%    \item This can be implemented in a database.
%
%    \item All parts of the webapplication must conform to this, use the 
%      centralised functions to enforce it.
%
%  \end{itemize}
%\end{frame}
%
%\begin{frame}
%  \begin{itemize}
%    \item Make sure to check both actions and resources.
%
%    \item Is this user allowed to edit personal settings?
%      Yes, he is, he's logged in.
%
%    \item Whose personal settings did the logged-in user edit?
%
%  \end{itemize}
%\end{frame}
%
%\begin{frame}
%  \begin{itemize}
%    \item When implementing the authorisation checks, watch out for ``positive 
%      authorisation''!
%
%    \item I.e.\ don't assume things will succeed, always write for default 
%      deny.
%
%    \item If your code is aborted, in any line, it should fail the 
%      authorisation.
%
%  \end{itemize}
%\end{frame}
%
%\subsection{Static Content}
%
%\begin{frame}
%  \begin{itemize}
%    \item Remember to protect files in the webserver's file system, e.g.\ PDFs 
%      and images if they are not publicly available.
%
%    \item Either set some ACLs for them to disallow fetching them using direct 
%      URLs.
%
%    \item Another possibility is to generate the content on the fly.
%
%  \end{itemize}
%\end{frame}
%
%\subsection{Don't trust the user}
%
%\begin{frame}
%  \begin{itemize}
%    \item Don't trust anything from the client-side.
%
%    \item Store all critical data on server-side, all access control related 
%      things should be on the server -- and check every time!
%
%  \end{itemize}
%\end{frame}
%

%%%%%%%%%%%%%%%%%%%%%%

\begin{frame}
  \small
  \printbibliography
\end{frame}

