%\documentclass[handout]{beamer}
\documentclass{beamer}
\usepackage[utf8]{inputenc}
\usepackage[T1]{fontenc}
\usepackage[ibycus,swedish,british]{babel}
\usepackage{url}
\usepackage{graphicx}
\usepackage{color}
\usepackage{subfig}
\usepackage{multicol}
\usepackage{amssymb,amsmath,amsthm}
\usepackage{booktabs}
\usepackage[squaren,binary]{SIunits}
\usepackage{verbatim}
\usepackage{listings}

\usepackage{xparse}
\ProvideDocumentEnvironment{exercise}{o}{%
  \setbeamercolor{block body}{bg=yellow!30,fg=black}
  \setbeamercolor{block title}{bg=yellow,fg=black}
  \IfValueTF{#1}{%
    \begin{block}{\translate{Exercise}: #1}
  }{%
    \begin{block}{\translate{Exercise}}
  }
}{%
  \end{block}
}

\mode<presentation>{%
  \usetheme{Berlin}
  \setbeamertemplate{footline}{%
    \begin{beamercolorbox}[colsep=1.5pt]{upper separation line foot}
    \end{beamercolorbox}
    \begin{beamercolorbox}[ht=2.5ex,dp=1.125ex,%
      leftskip=.3cm,rightskip=.3cm plus1fil]{author in head/foot}%
      \leavevmode{\usebeamerfont{author in head/foot}\insertshortauthor}%
      \hfill%
      {\usebeamerfont{institute in head/foot}%
        \usebeamercolor[fg]{institute in head/foot}\insertshortinstitute}%
    \end{beamercolorbox}%
    \begin{beamercolorbox}[ht=2.5ex,dp=1.125ex,%
      leftskip=.3cm,rightskip=.3cm plus1fil]{title in head/foot}%
      {\usebeamerfont{title in head/foot}\insertshorttitle}%
      \hfill\insertframenumber%
    \end{beamercolorbox}%
    \begin{beamercolorbox}[colsep=1.5pt]{lower separation line foot}
    \end{beamercolorbox}
  }
  \setbeamercovered{transparent}
  \setbeamertemplate{bibliography item}[text]
}

\usepackage[natbib,style=alphabetic,maxbibnames=99]{biblatex}
\addbibresource{overview.bib}

\title{%
  Side Channels and Covert Channels
}
\author{Daniel Bosk}
\institute[MIUN ICS]{%
  Department of Information and Communication Systems (ICS),\\
  Mid Sweden University, Sundsvall.
}
\date{\today}

\AtBeginSection[]{%
  \begin{frame}<beamer>
    \tableofcontents[currentsection]
  \end{frame}
}

\begin{document}

\begin{frame}
  \maketitle{}
\end{frame}


% Since this a solution template for a generic talk, very little can
% be said about how it should be structured. However, the talk length
% of between 15min and 45min and the theme suggest that you stick to
% the following rules:  

% - Exactly two or three sections (other than the summary).
% - At *most* three subsections per section.
% - Talk about 30s to 2min per frame. So there should be between about
%   15 and 30 frames, all told.


\section{Side Channels}

\subsection{Definition}

\begin{frame}
  \begin{definition}[Side Channel]
    A \emph{side channel} is an unintended channel emitting information which 
    is due to physical implementation flaws and not theoretical weaknesses or 
    forcing attempts.
  \end{definition}
  \begin{example}
    Using the standard algorithms for addition and multiplication (using the 
    binary number system) we can easily see that the time to perform \(3\times 
    25\) and \(7\times 25\) will be different.
  \end{example}
\end{frame}

\begin{frame}
  \begin{itemize}
    \item Looking at the numbers we have we see that \(3_{10} = 11_2\), 
      \(7_{10} = 111_2\) and \(25_{10} = 11001_2\)

    \item Assume each step in the algorithm takes one time unit.

    \item Then for \(11001\times 11\) we get:
      \begin{itemize}
        \item \(5\) time units for multiplying the last \(1\) in \(11\) with 
          each digit in \(11001\),

        \item another \(5\) time units for the next digit in \(11\),

        \item we have an additional \(1\) time unit for shifting the second 
          result one step,

        \item finally, we get \(6\) time units for adding the numbers.
      \end{itemize}

    \item For \(11001\times 111\) we get:
      \begin{itemize}
        \item \(5\) time units for each digit, hence \(15\) in total,

        \item we have two shifts, thus \(2\) time units more,

        \item finally we have \(7\) time units for adding.
      \end{itemize}
  \end{itemize}
\end{frame}

\begin{frame}
  \begin{itemize}
    \item Hence, the first multiplication takes \(17\) time units to perform 
      whereas the second takes \(24\) time units.

    \item This is called a timing attack and is one example of why 
      constant-time operations are desirable.

    \item However, in this example we cannot see the difference between 
      multiplication of \(2_{10} = 10_2\) and \(3_{10} = 11_2\).

    \item But in more complex situations this might not even be necessary.

  \end{itemize}
\end{frame}

\subsection{Timing Attacks}

\begin{frame}
  \begin{itemize}
    \item In~\cite{song2001timing} a timing attack on passwords sent over 
      encrypted SSH sessions was shown.

    \item As each keystroke in the password is sent in a separate package, the 
      attacker can observe the delay between keystrokes.

    \item They found that this gave a factor 50 advantage for guessing the 
      password.
  \end{itemize}
\end{frame}

\subsection{Acoustic Cryptanalysis}

\begin{frame}
  \begin{itemize}
    \item In~\cite{genkin2013rsa} the authors showed an attack to extract 
      a 4096-bit RSA private key from a laptop PC (GnuPG implementation of 
      RSA).

    \item Computers emit high-pitched noice during operation due to some of 
      their electronic components.

    \item This was used to derive the key used for decryption of some chosen 
      ciphertexts within an hour!

    \item Their results show that this attack can be accomplished by placing 
      a mobile phone (microphone) next to the target laptop.

    \item They also show a similar attack is possible by measuring the electric 
      potential of a computer's chassis, e.g.\ by just touching it.
  \end{itemize}
\end{frame}

\begin{frame}
  \begin{itemize}
    \item The acoustic signals are picked up from components in the power 
      supply.

    \item Individual CPU operations are too fast for a microphone to pick up.

    \item But long operations such as modular exponentiation (as in RSA) can 
      create a characteristic acoustic spectral signature which can be detected 
      using a microphone.
  \end{itemize}
\end{frame}


\section{Covert Channels}

\subsection{Definition}

\begin{frame}
  \begin{definition}[Covert Channel]
    A \emph{covert channel} is a mechanism that was not designed for 
    communication but which can nontheless be abused to allow information to 
    flow in a way which is not allowed in the security policy.
  \end{definition}
  \begin{definition}[Side Channel]
    A \emph{side channel} is an unintended channel emitting information which 
    is due to physical implementation flaws and not theoretical weaknesses or 
    forcing attempts.
  \end{definition}
\end{frame}

\begin{frame}
  \begin{itemize}
    \item The definitions do overlap.

    \item Usually one talks of side-channels in cryptography and 
      covert-channels in larger systems.
  \end{itemize}
\end{frame}

\subsection{Bell-LaPadula}

\begin{frame}
  \begin{itemize}
    \item BLP says ``no read up'' and ``no write down''.

    \item What happens if I try to ``write up'' but something already exists?

    \item Using this you can create a covert channel.

    \item Each denied operation is one bit of information (entropy) revealed by 
      the security mechanisms.

  \end{itemize}
\end{frame}

\begin{frame}
  \begin{itemize}
    \item The Naval Research Laboratory invented the NRL-Pump.

    \item This is a device used to limit the bandwidth of possible covert 
      channels.

    \item The pump allows flow upwards.

    \item But we need some flow downwards too, e.g.\ acknowledgement that data 
      was received correctly.

    \item Bandwidth of possible covert channels are limited using buffers and 
      randomised timing of acknowledgements among other things.
  \end{itemize}
\end{frame}

\begin{frame}
  \begin{example}[Logistics system]
    \begin{itemize}
      \item A military warehouse holds classified equipment, but the warehouse 
        itself it not classified.

      \item A person in the logistics department doesn't have sufficient 
        clearance.

      \item What happens when this person wants to use the space for other 
        things?

      \item Make some things up and put in there so it looks occupied.

      \item What if this person needs some of the items in the cover story?

    \end{itemize}
  \end{example}
\end{frame}

\subsection{Page Faults and LEDs}

\begin{frame}
  \begin{itemize}
    \item Another example of how a covert channel might be constructed is page 
      faults.

    \item What if we manage to place things in memory in such a way that it 
      extends into another page.

    \item What if that page is not in memory?

    \item Then we know from either measuring time (we notice if a page-fault 
      occurs) or obsering the disk activity.
  \end{itemize}
\end{frame}

\begin{frame}
  \begin{itemize}
    \item Yet another example is the LEDs indicating disk activity.

    \item If this LED is connected to the serial lines, indicating when data is 
      sent, then information about the data is leaked.

    \item Further, the electronic components in computer displays leak radio 
      signals caused by the states of pixels etc.

    \item There has been shown that you can pick up the picture of the screen 
      from these signals two rooms away.

  \end{itemize}
\end{frame}

\subsection{Cheating on the Exam}

\begin{frame}
  \begin{itemize}
    \item Which times someone goes to the toilet: if it is an even minute it's 
      a one, if odd it's a zero.

    \item The rythm someone clicks their pen against the desk: a change in 
      rythm is a one.
      However, this needs some synchronisation.

    \item Drum Morse code on the table.

    \item \dots

  \end{itemize}
\end{frame}


%%%%%%%%%%%%%%%%%%%%%%

\begin{frame}{Referenser}
  \small
  \printbibliography{}
\end{frame}

\end{document}
